\documentclass{article}
\usepackage[utf8]{inputenc}

\title{Computer History- Memory and Storage}
%add your names to the author part
\author{Garrett Wood,Justin Urbanich }
\date{October 2020}

\begin{document}

\maketitle

\section{Introduction}
Ever since computers first started to become widely used after WW2, the need to store, manage, and update data has grown. Mostly starting with simple paper punchcards that "stored" programs to be run, memory and storage have evolved over time to combat the ever increasing size of programs in the modern day.
\section{Time Period}
The first real instance of main memory in computers took the form of "delay lines" which actually predate the first real electronic computers,being developed around 1920. This was quickly overshadowed however in the 1930's by magnetic drum memory which can be considered the precursor to the standard hard drive we see today. This too was overtaken by magnetic core memory in 1955 which was the first ever form of random access memory. However, it was not dense enough to be reasonably upscaled, and its difficulty to manufacture assured that semiconductor based memory would replace it in the 1960's. Storage has also undergone many changes throughout the years. 
The first form of mass storage for computers was magnetic tape, which like hard drives today was composed of a magnetically coated surface that was moved over a seeker head that interpreted data from the magnetization of the film. The next major step in mass storage was magnetic disk drives, commonly known today as hard drives. These were however, very large and expensive at the time meaning that they couldnt really be used in the commercial sector so floppy disks were what people started using. Nowadays, hard drives have become the cheapest and most effective way to store large amounts of data. While things like tape still exist and solid state drives are becoming faster and larger capacity, most data is still stored on hard drives.

\section{Computer Hardware}
There are many hardware components to memory and storage. Magnetic tapes are fairly cheap for the amount of storage they offer as they are fairly easy to produce. However due to their fairly straightforward and physical approach to storing data magnetic tapes are fairly slow to access memory. However the main downside of magnetic tapes are their size, despite a cheap cost for the amount of data they can store magnetic tapes are very large relitive to how much data they can store. This is a large drawback as storing large amounts of data on magnetic tapes requires a large amount of space.
Another type of hardware developed for data storage is the punched card. Punched cards are pieces of material, usually paper like, that have data stored on them by punching holes in the paper that convay the data. This form of storage works by converting data into a form that can be convayed through punched holes and then punching those holes. This means the data is stored physically and can only be read by a machine designed to read punched cards. This form of data storage has many downsides such as the need for specialised machines to write and read the data, the fact the punched cards can easily become distorted making the data on them hard or even impossible to read, and the fact that punched card storage is a purely physical type of storage making performing certain tasks require it to be converted into digital data that can be modified by a computer. Lastly the data cannot be modified on a punched card without requiring another card. On a related note without the cards themselves there is no way to store the data at all.
One of the most common forms of hardware that is specific to memory and storage is the CD/DVD. CDs and DVDs are disks that have data encoded into them. The difference between them is that CDs cannot store video while DVDs can. This form of storage is vastly superior to the prior two as discs are small, cheap to produce, can store much more data, access data faster, can be reused and while they do require certain things to be written and read they do not require specific machines to read and write them. Discs are made of plastic and are extremely small, they work by encoding the data into the surface of the disc. This does mean that they are suceptible to becoming unreadable if the disc itself is damaged however for the most part discs are a very reliable way to store data. Discs can be used to store many forms of data from software to photo to files to audio. Discs can also be reused, while somewhat limited due to the ways discs store data for the most part data on a disc can be taken off the disc and replaced, updated, modifed or added to without much hastle.
Magnetic discs are another way to store data. Magnetic discs store data using magnets. This is done by applying a specific magnetic force to the disc that can be read by specifically designed hardware. Magnetic discs are somewhat of a mix between magnetic tapes and DVDs/CDs as magnetic discs are discs but are used more as a better version of magnetic tapes. This is because they are much more specialised, few devices not specifcally designed to read magnetic discs can do so. The cost of magnetic discs is also higher then plastic discs but magnetic discs are more resiliant to damage that would cause the data to be inaccessible.
The last 4 peices of hardware have been storage hardware, or hardware that stores data. However this is different from memory which is the current process and actions of a computer. Without memory computers would not be able to interact with stored data, doing nearly anything on a computer uses memory as functions must be run to make a computer function. The parts of a computer that run these functions that make the computer work and allow it to do anything are part of the memory. The actions of a computer are undertaken by a CPU or computer processing unit. This is what turns input into instructions for the computer and executes them. A CPU is comprised of two main hardware components, the semiconductor and the magnetic core.
Magneitic cores are usually small chip like objects. They are covered in tiny metal threads that connect all the various parts of the core. Each tiny part of the core can either be magnetised or not. The sequence of the parts of the core that are magnetised and not is essentiallyy read as 1s and 0s by the computer. This pattern of 1s and 0s is called binary and is the most simple form of any instructions given to the computer. After being given instructions on what to do the core will (after many processing steps) convert the instructions into binary and magnetise and demagnetise the parts of the core accordingly. Typically computers have 1 core however as time has gone on the amount of pins and circuts on cores have increased dramatically, this means a computer can do more things at one, executing more nstructions at once and completing the task in less time as a result. Also some pieces of software require a certain amount of bits for a core to have in order to be able to run. This is because these softwares need to be able to perform certain functions at certain times, including at the same time and smaller cores will not be able to do so because they might only have enough bits to perform 1 function at a time.
Semiconductor memory is a more recent type of memory. It is usually more powerful then core based memory leading to it being preferred in many devices. Semicondutor memory is more powerful because instead of precessing data through magnetising tiny pins it uses semiconductive metals to store data in a chemical form. This is not directly faster however due to the density at which these semiconductive metals can be semiconductor memory is able to compact many tiny pins worth of bits into a much smaller space.
\section{Computer Software}

\section{Conclusion}

\section{References}
https://www.computerhistory.org/revolution/memory-storage/\\
https://en.wikipedia.org/wiki/Drum\_memory\\
https://en.wikipedia.org/wiki/Magnetic-core\_memory\\
https://en.wikipedia.org/wiki/Floppy\_disk\\
https://www.vskills.in/certification/tutorial/data-storage-devices/\\
\end{document}
