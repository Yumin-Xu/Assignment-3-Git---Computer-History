\documentclass{article}
\usepackage[utf8]{inputenc}

\title{Computer History- Memory and Storage}
%add your names to the author part
\author{Garrett Wood, }
\date{October 2020}

\begin{document}

\maketitle

\section{Introduction}
Ever since computers first started to become widely used after WW2, the need to store, manage, and update data has grown. Mostly starting with simple paper punchcards that "stored" programs to be run, memory and storage have evolved over time to combat the ever increasing size of programs in the modern day.
\section{Time Period}
The first real instance of main memory in computers took the form of "delay lines" which actually predate the first real electronic computers,being developed around 1920. This was quickly overshadowed however in the 1930's by magnetic drum memory which can be considered the precursor to the standard hard drive we see today. This too was overtaken by magnetic core memory in 1955 which was the first ever form of random access memory. However, it was not dense enough to be reasonably upscaled, and its difficulty to manufacture assured that semiconductor based memory would replace it in the 1960's. Storage has also undergone many changes throughout the years. 
The first form of mass storage for computers was magnetic tape, which like hard drives today was composed of a magnetically coated surface that was moved over a seeker head that interpreted data from the magnetization of the film. The next major step in mass storage was magnetic disk drives, commonly known today as hard drives. These were however, very large and expensive at the time meaning that they couldnt really be used in the commercial sector so floppy disks were what people started using. Nowadays, hard drives have become the cheapest and most effective way to store large amounts of data. While things like tape still exist and solid state drives are becoming faster and larger capacity, most data is still stored on hard drives.

\section{Computer Hardware}

\section{Computer Software}

\section{Conclusion}

\section{References}
https://www.computerhistory.org/revolution/memory-storage/\\
https://en.wikipedia.org/wiki/Drum\_memory\\
https://en.wikipedia.org/wiki/Magnetic-core\_memory\\
https://en.wikipedia.org/wiki/Floppy\_disk\\
https://www.vskills.in/certification/tutorial/data-storage-devices/\\
\end{document}
